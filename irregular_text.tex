\documentclass[varwidth=true, border=10mm]{standalone}
\usepackage{fontspec}
\setmainfont[Scale=2.0]{ML4Science}  % Increased Font Size

\usepackage{tikz}
\usetikzlibrary{calc}

% Command for irregular word placement
\newcommand{\irregularword}[1]{% #1: Text
  \pgfmathsetmacro{\yshift}{(random()-0.5)*3} % Random y-shift between -3pt and 3pt
  \pgfmathsetmacro{\rotation}{(random()-0.5)*10} % Random rotation between -5° and 5°
  \tikz[baseline]{
    \node[inner sep=0pt, outer sep=0pt, anchor=base, yshift=\yshift pt, rotate=\rotation] (text) {\strut #1};
  }%
}

\usepackage{xparse}
\ExplSyntaxOn
\NewDocumentCommand{\processtext}{+m}{
  \seq_set_split:Nnn \l_tmpa_seq { ~ } { #1 }
  \seq_map_inline:Nn \l_tmpa_seq { \irregularword{##1} }
}
\ExplSyntaxOff

\begin{document}

\section*{Demonstration of Strikes}

\processtext{
This is a demonstration of different strike-through styles in a single paragraph.
Sometimes, simple strikes are enough to cross out certain words or phrases,
but other times you may want a wiggly line for a more dynamic effect.
For a bold, artistic touch, the brush-like style adds a creative feel.
Finally, when everything needs to connect seamlessly, the connected line strike provides a polished and professional look.
We do not have a strike command yet, but we can create one.
52562 46152 + 4 = 0
}


\end{document}